\documentclass[20pt,margin=1in,innermargin=-4.5in,blockverticalspace=-0.25in]{tikzposter}
\geometry{paperwidth=42in,paperheight=30in}
\usepackage[utf8]{inputenc}
\usepackage{amsmath}
\usepackage{amsfonts}
\usepackage{amsthm}
\usepackage{amssymb}
\usepackage{mathrsfs}
\usepackage{graphicx}
\usepackage{adjustbox}
\usepackage{enumitem}
\usepackage[backend=biber,style=numeric]{biblatex}
\usepackage{emory-theme}

\usepackage{mwe} % for placeholder images
\usepackage{biblatex}
\addbibresource{bibliography.bib}

% set theme parameters
\tikzposterlatexaffectionproofoff
\usetheme{EmoryTheme}
\usecolorstyle{EmoryStyle}

\title{Contextual Text Style Transfer}
\author{P. Champion, J.-B. Gaya and M. Herrera-Martinez}
\institute{Sorbonne Université}
%\titlegraphic{\includegraphics[width=0.07\textwidth]{Emory_vt_280.png}}

% begin document
\begin{document}
\maketitle
\centering
\begin{columns}
    \column{0.32}
    \block{Motivation \& Background }{
    Text style transfer consists in the translation of a sentence into a desired style. It encompasses various applications, e.g. sentiment manipulation and formalized writing. 
    
    Previous work relied mostly on parallel corpora with a sentence-to-sentence  learning framework \cite{bahdanau2014neural}.
    In the studied paper, an approach to perform this task while maintaining the sentence coherent with its surrounding context is discussed; hence the name contextual text style transfer. 
    
    This poster summarizes the work conducted based on this paper. The section \ref{ModelSection} introduces the key concepts and modules used to build the model, and the section \ref{ResultsSection}
    
    
    
    }
    \block{Model}{
    \label{ModelSection}
    %\section{Key Concepts}
    \section{Dataset}
    
    
    
    }
    \column{0.36}
    \block{Results}{
    \label{ResultsSection}
    }

    \column{0.32}

    
    \block{Remarks}{

    }
    

    \block{References}{
        \vspace{-1em}
        \begin{footnotesize}
        \printbibliography[heading=none]
        \end{footnotesize}
    }

\end{columns}
%\bibliographystyle{plain}
%\bibliography{refs}
\end{document}